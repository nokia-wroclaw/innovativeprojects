%%%%%%%%%%%%%%%%%%%%%%%%%%%%%%%%%%%%%%%%%%%%%%%%%%%%%%%%%%%%%%%%%%%%%%%%%%%%%%%%%%%%%%
\begin{project}
{Projects Map}
{Web Application that allows to create map of projects that are developed in given department/company. Projects should be described by: short description, technologies, list of developers etc. Each developer should  be described by list of technologies/frameworks that they know - that will allow to get help in given topic by others developers.} 
{
Features:
\begin{itemize}
	\item Drawing map of office with projects/developers
	\item Adding/editing projects/developers
	\item Hierarchy view of department/company
	\item Adding/editing department/company
\end{itemize}
}
{Basic knowledge about Javascript}
{Mateusz Wierzbicki}
{1 semester}
{2-3}
\end{project}
%%%%%%%%%%%%%%%%%%%%%%%%%%%%%%%%%%%%%%%%%%%%%%%%%%%%%%%%%%%%%%%%%%%%%%%%%%%%%%%%%%%%%%
\begin{project}
{Nokia integration game}
{Corporation version of "Time's up" game for mobile phones with centralized DB. One part of app is web application which allow to add custom characters to game. Second part is game for mobiles. Game ask backed for random set of characters and leading 4 rounds of game (description, one word, showing without speaking and pose) - like in original "Time's up" game.} 
{
Web application:
\begin{itemize}
	\item List of collections
	\item Managing user collections of characters (adding, editing, exporting, importing, tagging)
	\item API for mobile app
	\item Downloading random set of characters from chosen collection
	\item Downloading random set of characters for specific characters tags (e.g. \#sport, \#fantasy)
	\item Adding new tags to characters
\end{itemize}
\bigbreak
Mobile application:
\begin{itemize}
	\item Downloading characters from webapp
	\item Showing list of characters and possibility to reject/exchange a few of them
	\item Gameplay (4 round, 2 teams) with counting down time, points and displaying rules of each round
\end{itemize}
}
{
\begin{itemize}
	\item Basic of JavaScript,
	\item Be open to learning mobile technologies like: Ionic, React Native, etc.
\end{itemize}
}
{Kamil Mleczko}
{1 semester}
{2-3}
\end{project}
%%%%%%%%%%%%%%%%%%%%%%%%%%%%%%%%%%%%%%%%%%%%%%%%%%%%%%%%%%%%%%%%%%%%%%%%%%%%%%%%%%%%%%
\begin{project}
{Developers dashboard}
{
Application allows creating dashboards with information about important things for developers like result of builds in CIs systems. Dashboard contains tiles with results and is customizable via web interface. Sources should be connectable via plugins. Plugin is a piece of code which contains fetching data, mapping fetched data to results and presenting result on tiles.  
\bigbreak
Target of the project is to run addtional computer which presents for all developers dashboard with project development status.
}
{
Features:
\begin{itemize}
	\item Dashboard with tiles
	\item Configuration of dashboard via web app
	\item Sources connectable via plugins
	\item Notification about events (mail, slack)
	\item Static and dynamic tiles (for example develop branch and feature builds)
\end{itemize}
}
{Basic knowledge about Javascript}
{Mateusz Sikora}
{1 semester}
{2-4}
\end{project}
%%%%%%%%%%%%%%%%%%%%%%%%%%%%%%%%%%%%%%%%%%%%%%%%%%%%%%%%%%%%%%%%%%%%%%%%%%%%%%%%%%%%%%
\begin{project}
{Mailing groups browser}
{Application subscribes to mailing group via email (like normal user) and aggregates recived mails to threads. Threads should be searchable and filterable in the frontend part of application.}
{
Features:
\begin{itemize}
	\item Mailing group client which parses mails, aggregates and persists them in DB
	\item API for data
	\item Client side for browsing, filtering, searching and possibility to contact with author of threads
	\item Personalized settings for spam filters and searching
\end{itemize}
}
{Basic knowledge about Javascript}
{Mateusz Sikora}
{1 semester}
{2-4}
\end{project}
%%%%%%%%%%%%%%%%%%%%%%%%%%%%%%%%%%%%%%%%%%%%%%%%%%%%%%%%%%%%%%%%%%%%%%%%%%%%%%%%%%%%%%
\begin{project}
{Comparing graph databases}
{Based on prepared dataset that describes relations between ancestors (family tree) you will have to present those relations in a tree form, store and transform them using graph databases:
\begin{itemize}
	\item OrientDB
	\item HGraphDB
\end{itemize}
As a conclusion you should compare those two databases based on performance and convenience for that task.
}
{
Following project includes::
\begin{itemize}
	\item Storing and presenting relation data in tree form in graph databases
	\item Scripts that perform transformations on the data, such as:
		\begin{itemize}
			\item retrieve n-th ancestor/child based on relation column
			\item filter children based on column value
			\item get all elements with given ancestor
		\end{itemize}
\end{itemize}
}
{
\begin{itemize}
	\item Basic knowledge about databases
	\item Basic knowledge about data structures
	\item Willing to learn new technologies
\end{itemize}
}
{Filip Płotnicki}
{1 semester}
{2-4}
\end{project}
%%%%%%%%%%%%%%%%%%%%%%%%%%%%%%%%%%%%%%%%%%%%%%%%%%%%%%%%%%%%%%%%%%%%%%%%%%%%%%%%%%%%%%
\begin{project}
{Comparing map-reduce methods}
{
Based on prepared dataset that describes relations between ancestors (family tree) you will have to present those relations in a tree form and store in MongoDB. Additionally you should be able to transform them using two methods:
\begin{itemize}
	\item default map-reduce mechanism in MongoDB
	\item Spark connector for MongoDB
\end{itemize}
As a conclusion you should compare those two methods based on performance and convenience for that task.
}
{
Following project includes::
\begin{itemize}
	\item Storing and presenting relation data in tree form in MongoDB
	\item Transformations on the data using default map-reduce and Spark connector:
		\begin{itemize}
			\item retrieve n-th ancestor/child based on relation column
			\item filter children based on column value
			\item get all elements with given ancestor
	\end{itemize}
\end{itemize}
}
{
\begin{itemize}
	\item Basic knowledge about databases (MongoDB)
	\item Basic knowledge about data structures
	\item Willing to learn new technologies (Spark)
\end{itemize}
}
{Krzysztof Grining}
{1 semester}
{2-4}
\end{project}
%%%%%%%%%%%%%%%%%%%%%%%%%%%%%%%%%%%%%%%%%%%%%%%%%%%%%%%%%%%%%%%%%%%%%%%%%%%%%%%%%%%%%%
\begin{project}
{Converter for table-based data to trees}
{Based on prepared dataset that describes relations between ancestors (family tree) stored in a flat table you will have to prepare a "converter" that transforms the data in the flat table to a tree structure, which should be stored in Hbase. You should be able to perform transformations on the stored tree. You are free to choose or come up with a method for generating and storing the trees.}
{
Following project includes::
\begin{itemize}
	\item Converter script/application that converts flat table data into tree structured data
	\item Script that performs transformations on the tree-structured data
		\begin{itemize}
			\item retrieve n-th ancestor/child based on relation column
			\item filter children based on column value
			\item get all elements with given ancestor
		\end{itemize}
\end{itemize}
}
{
\begin{itemize}
	\item Basic knowledge about distributed computing and databases
	\item Basic knowledge about data structures
	\item Willing to learn new technologies
\end{itemize}
}
{Filip Płotnicki}
{1 semester}
{2-4}
\end{project}
%%%%%%%%%%%%%%%%%%%%%%%%%%%%%%%%%%%%%%%%%%%%%%%%%%%%%%%%%%%%%%%%%%%%%%%%%%%%%%%%%%%%%%
\begin{project}
{Web component with Tanks game}
{The goal of project is to preare a React component, that can be included on webpage and after pressing a certain combination of keys it overlays the webpage and opens up a game based on "Tanks" (see: \url{https://pl.wikipedia.org/wiki/Battle_City}).}
{
Following project includes::
\begin{itemize}
	\item React component with playable game of Tanks
	\item Basic gameplay features: driving the tank, shooting at the enemies, collisions, destructible terrain
	\item Simple AI for enemies
	\item Highscores
\end{itemize}
 You can add your own features and ideas in the game.
}
{
\begin{itemize}
	\item Basic knowledge about JavaScript
	\item Basic knowledge about WebGL
	\item Willing to learn new technologies
\end{itemize}
}
{Andrzej Rozenfeld}
{1 semester}
{2-4}
\end{project}
%%%%%%%%%%%%%%%%%%%%%%%%%%%%%%%%%%%%%%%%%%%%%%%%%%%%%%%%%%%%%%%%%%%%%%%%%%%%%%%%%%%%%%
\begin{project}
{APT plugin for Nexus}
{The goal of this project is to prepare a plugin for Nexus repository application (\url{https://www.sonatype.com/nexus-repository-sonatype}) for managing APT repositories.}
{
Following project includes::
\begin{itemize}
	\item Plugin that adds functionalities to Nexus:
		\begin{itemize}
			\item creating/removing/modifying APT repositories
			\item APT repository management through Web GUI
		\end{itemize}
\end{itemize}
}
{
\begin{itemize}
	\item Basic knowledge about Java
	\item Basic knowledge about APT repositories
	\item Willing to learn new technologies
\end{itemize}
}
{Mateusz Stanuch}
{1 semester}
{2-4}
\end{project}
%%%%%%%%%%%%%%%%%%%%%%%%%%%%%%%%%%%%%%%%%%%%%%%%%%%%%%%%%%%%%%%%%%%%%%%%%%%%%%%%%%%%%%
\begin{project}
{Donation application}
{
Donation platfrom based on constant contact between both sides (receivers and donors). Users are able to support receivers with payments and they receive information about some progress etc. Donors use mobile application where they can see messages from receivers, list of payments and donate using external payments system like PayU. Receivers use web application for monitoring donations and sending messages to donors.
 \bigbreak
*Receivers are understood as some foundation, startup etc.
} 
{
Web application:
\begin{itemize}
	\item List of payments
	\item Sending messages to all or selected users
\end{itemize}
\bigbreak
Mobile application:
\begin{itemize}
	\item Payments (test of services like dotpay, PayU, PayPal, etc.)
	\item Receiving messages, notifications etc.
\end{itemize}
Final scope of project will be set with the team.
}
{
\begin{itemize}
	\item Basic knowledge about Android
	\item Basic knowledge about Web programming
	\item Willing to learn new technologies
\end{itemize}
}
{Ewa Kaczmarek}
{1 semester}
{3-4}
\end{project}
%%%%%%%%%%%%%%%%%%%%%%%%%%%%%%%%%%%%%%%%%%%%%%%%%%%%%%%%%%%%%%%%%%%%%%%%%%%%%%%%%%%%%%
\begin{project}
{Recruitment application}
{Mobile application on Android to support job fairs with web application for management. Tablets are taken to job fairs where candidates can fill the form for selected job offers. All the applications are presented then in web application where recuiters can see the list of candidates and contact with them via mail. List of job offers can be changed between different job fairs. Some statistics should be provided to compare job fairs and job offers interest.}
{
Web application:
\begin{itemize}
	\item List of job offerts
	\item List of applications for selected job offers
	\item Create new events
	\item Create new job offerts for events
	\item Statistics (how many candidates on specific event applied on selected job offer)
	\item Sending mails to one or more cadidates
\end{itemize}
\bigbreak
Mobile application:
\begin{itemize}
	\item Present job offers
	\item Simple form per job offer
	\item Work in offline mode
	\item Send forms when online
\end{itemize}
Final scope of project will be set with the team.
}
{
\begin{itemize}
	\item Basic knowledge about Android
	\item Basic knowledge about Web programming
	\item Willing to learn new technologies
\end{itemize}
}
{Ewa Kaczmarek}
{1 semester}
{3-4}
\end{project}
%%%%%%%%%%%%%%%%%%%%%%%%%%%%%%%%%%%%%%%%%%%%%%%%%%%%%%%%%%%%%%%%%%%%%%%%%%%%%%%%%%%%%%
\begin{project}
{UI issue feedback}
{A Chrome (web browser) extension or web application for finding and selecting those parts of web application (website) which are considered as ugly, bugged or defected.}
{
Following project includes: 
\begin{itemize}
	\item An extension or web application for giving feedback about unliked part of application with a visual preview (an image or live) of that part (or the entire page with those parts selected). 
	\item A control panel(also web application) where those feedbacks are stored and managed. 
\end{itemize}
\bigbreak
Developing applications by group of developers comes with troubles with making an agreement of visual aspects or functionality of an app. Writing e-mails and describing something using only text consume too much time and sometimes just doesn't work, specially if one feature has more than one author. Gathering feedbacks from many sources is also hard. 
\bigbreak
Project described above makes this whole process faster, easier and much cleaner, specially for someone who is responsible for fixing. 
}
{
\begin{itemize}
	\item Basic knowledge about any web programming language (and optionally creating Chrome extensions) and any database system.
	\item Willing to learn new technologies
\end{itemize}
}
{Maciej Bakowicz}
{1 semester}
{2-4}
\end{project}
%%%%%%%%%%%%%%%%%%%%%%%%%%%%%%%%%%%%%%%%%%%%%%%%%%%%%%%%%%%%%%%%%%%%%%%%%%%%%%%%%%%%%%
\begin{project}
{NERD - NEwcomer Request Delivery}
{Develop a tool that will speed up / automate newcomer enablement process by adding new user to projects, sending mail requests or manuals, and creating Jira tickets via REST API.} 
{
Manager wants to give access rights to tools required to work with project for every member that joins development team (Jenkins, application server, Jira, confluence, GIT repository, etc). There also must be mail notification aimed to development team and product owner about new team member. Newcomer should get mail with set of instructions/manuals/requrements for quick start-up.
\bigbreak
There are two main use scenarios:
\begin{itemize}
	\item Project configuration in NERD Tool via web gui:
		\begin{itemize}
			\item Setup of messages to be sent on new team member arrival
				\begin{itemize}
					\item Configuration of message/content. Mail message template can be edited with BBCode or Markdown.
					\item Configuration of list of recepients 
				\end{itemize}
			\item Configuration of issues to be created on Jira service - target project, required fields etc
		\end{itemize}
	\item New team member arrival - particular group of people can trigger actions configured in previous step for newcomer
\end{itemize}
}
{
Minimal experience or eager to learn:
\begin{itemize}
	\item Some language for backend logic (i.e. Java + Play Framework)
	\item Some language for frontend logic (i.e. JavaScript + Angular2 or Java + Vaadin)
	\item Database knowledge
	\item RESTful web services
\end{itemize}
}
{Blazej Krystek}
{1 semester}
{3-5}
\end{project}
%%%%%%%%%%%%%%%%%%%%%%%%%%%%%%%%%%%%%%%%%%%%%%%%%%%%%%%%%%%%%%%%%%%%%%%%%%%%%%%%%%%%%%
\begin{project}
{Cross application notification system}
{Implement platform allowing for easy management and aggregation of users notifications. Service should collect notifications from multiple applications and/or users. Platform should distribute notifications to subscribed end users. Additionally, there should be embeddable web component capable to displaying all unread user notification.} 
{
\begin{enumerate}
	\item Web component should allow for:
		\begin{itemize}
			\item easy embed inside external applications
			\item display aggregated notifications
			\item dismiss single/all notification
			\item show details and links
			\item manage subscribed notification sources and channels
		\end{itemize}
	\item Service should:
		\begin{itemize}
			\item be secured source of data for web component
			\item provide API for automatic notifications from applications
			\item provide way to create manual notifications
			\item allow scope notification message by type (info/warning/error), applications, topic and user/user groups
			\item create easy way to notify end user about not read messages
			\item allow for scale up for high-traffic
		\end{itemize}
\end{enumerate}
}
{
\begin{itemize}
	\item Any programming language
	\item Base web technologies knowledge
	\item Any DB system knowledge
	\item Eager to learn new technologies
\end{itemize}
}
{Dominik Markiewicz}
{1 semester}
{2-6}
\end{project}
%%%%%%%%%%%%%%%%%%%%%%%%%%%%%%%%%%%%%%%%%%%%%%%%%%%%%%%%%%%%%%%%%%%%%%%%%%%%%%%%%%%%%%
\begin{project}
{Cross-applications shortcuts as a web component}
{When many web services are operated and advertised by one entity (department, company, whatever) it is wise to have consistent way to easily move user bwetween applications. Good example are Google web apps or Microsoft web apps, where it's always obvious how to jump between services in given company portfolio - by using same looking shortcuts button in every application. The goal of the project is to have web-based service that would allow for creation, maangement and display of such common component for consistent linking to many web applicatiions/pages.} 
{
Minimal finished project allows for:
\begin{itemize}
	\item Separate web application where one can 
		\begin{itemize}
			\item create new apps - with their icons and links
			\item order or position of particlar application on applications list
		\end{itemize}
	\item Web component in any technology, that can be embedded in navbar of any application, and when clicked will display list of applications user can jump to with clickable links/anchors.
\end{itemize}
\bigbreak
Possible extension: created app could monitor health of linked applications and disable/enable or modify view of the links displayed depending on the status of linked application (unresponsive, maintanance or similar).
}
{
\begin{itemize}
	\item Any programming language
	\item Web technologies knowledge
	\item Any DB system knowledge
	\item Eager to learn new technologies
\end{itemize}
}
{Mateusz Wronski, Dominik Markiewicz}
{1 semester}
{4}
\end{project}