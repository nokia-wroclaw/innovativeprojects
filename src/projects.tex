%%%%%%%%%%%%%%%%%%%%%%%%%%%%%%%%%%%%%%%%%%%%%%%%%%%%%%%%%%%%%%%%%%%%%%%%%%%%%%%%%%%%%%
\begin{project}
{Tools portfolio}
{Web application with tools presentation as tiles with some description.} 
{
Application should allow management of several tools and present them in a nice layout.
\bigbreak
Proposal functionalities:
\begin{enumerate}
	\item Each tool represented as a tile - image, title, short description and url
	\item Tool details on modal (long description, repository etc.)
	\item Filtering - simple filters based on tags
	\item Add to favourites
	\item Notifications
	\item Permissions - different views and details for different users
	\item Admin panel - content management only for admins
\end{enumerate}

Final scope of project will be set with the team.
}
{Nice to have basic JS knowledge}
{Ewa Kaczmarek}
{1 semester}
{4-6}
\end{project}
%%%%%%%%%%%%%%%%%%%%%%%%%%%%%%%%%%%%%%%%%%%%%%%%%%%%%%%%%%%%%%%%%%%%%%%%%%%%%%%%%%%%%%
\begin{project}
{Workshop environment}
{Web application for creating programming workshops.} 
{
Main goal of application is to support programming workshops. Application should allow workshop creation dedicated for any programming language.
\bigbreak
Proposal functionalities:
\begin{enumerate}
	\item Working examples - code with explanations and working results.
	\item Syntax highlighting.
	\item Splitting material into chapters - in each chapter we extend code from the previous one (it is like building application in steps).
	\item Showing tasks for participants with timer at the end of each timer.
\end{enumerate}
 
Final scope of project will be set with the team.
}
{Web technologies knowledge}
{Ewa Kaczmarek}
{1 semester}
{3-4}
\end{project}
%%%%%%%%%%%%%%%%%%%%%%%%%%%%%%%%%%%%%%%%%%%%%%%%%%%%%%%%%%%%%%%%%%%%%%%%%%%%%%%%%%%%%%
\begin{project}
{Retro Tool}
{Web application for supporting Retrospective Meetings.}
{
Retrospective Meeting is part of Scrum methodology. During this meeting, team reflects the past Sprint, discuss on improvements, identifies good practices. The result of the Retrospective is set of Working Agreements (set of rules/disciplines/processes the team agrees to follow) and Action Items (small task assigned to team member to be done before next Retrospective Meetings).
\bigbreak
Application should allow management of several projects and create new posts in each of them and also show board with Working Agreements and Action Items.
\bigbreak
Proposal functionalities:
\begin{enumerate}
	\item General view (only for admins):
		\begin{enumerate}
			\item Projects management - adding/editing/removing projects, assigning members to them
			\item Users management - adding/editing/removing users
		\end{enumerate}
	\item Single project view (available only for assigned users):
		\begin{enumerate}
			\item Main dashboard with list of open Action Items and all Working Agreements set by the team
			\item Board with availability of posts creation.
			\item Post can have different category (new idea, code improvement etc.).
			\item Adding comments to posts.
		\end{enumerate}
\end{enumerate}
}
{Nice to have basic JS knowledge}
{Ewa Kaczmarek}
{1 semester}
{4-5}
\end{project}
%%%%%%%%%%%%%%%%%%%%%%%%%%%%%%%%%%%%%%%%%%%%%%%%%%%%%%%%%%%%%%%%%%%%%%%%%%%%%%%%%%%%%%
\begin{project}
{NERD - NEwcomer Request Delivery}
{Develop a tool that will speed up / automate newcomer enablement process by adding new user to projects, sending mail requests or manuals, and creating Jira tickets via REST API.} 
{
Manager wants to give access rights to tools required to work with project for every member that joins development team (Jenkins, application server, Jira, confluence, GIT repository, etc). There also must be mail notification aimed to development team and product owner about new team member. Newcomer should get mail with set of instructions/manuals/requrements for quick start-up.
\bigbreak
There are two main use scenarios:
\begin{itemize}
	\item Project configuration in NERD Tool via web gui:
		\begin{itemize}
			\item Setup of messages to be sent on new team member arrival
				\begin{itemize}
					\item Configuration of message/content. Mail message template can be edited with BBCode or Markdown.
					\item Configuration of list of recepients 
				\end{itemize}
			\item Configuration of issues to be created on Jira service - target project, required fields etc
		\end{itemize}
	\item New team member arrival - particular group of people can trigger actions configured in previous step for newcomer
\end{itemize}
}
{
Minimal experience or eager to learn:
\begin{itemize}
	\item Some language for backend logic (i.e. Java + Play Framework)
	\item Some language for frontend logic (i.e. JavaScript + Angular2 or Java + Vaadin)
	\item Database knowledge
	\item RESTful web services
\end{itemize}
}
{Blazej Krystek}
{1 semester}
{3-5}
\end{project}
%%%%%%%%%%%%%%%%%%%%%%%%%%%%%%%%%%%%%%%%%%%%%%%%%%%%%%%%%%%%%%%%%%%%%%%%%%%%%%%%%%%%%%
\begin{project}
{Cross application notification system}
{Implement platform allowing for easy management and aggregation of users notifications. Service should collect notifications from multiple applications and/or users. Platform should distribute notifications to subscribed end users. Additionally, there should be embeddable web component capable to displaying all unread user notification.} 
{
\begin{enumerate}
	\item Web component should allow for:
		\begin{itemize}
			\item easy embed inside external applications
			\item display aggregated notifications
			\item dismiss single/all notification
			\item show details and links
			\item manage subscribed notification sources and channels
		\end{itemize}
	\item Service should:
		\begin{itemize}
			\item be secured source of data for web component
			\item provide API for automatic notifications from applications
			\item provide way to create manual notifications
			\item allow scope notification message by type (info/warning/error), applications, topic and user/user groups
			\item create easy way to notify end user about not read messages
			\item allow for scale up for high-traffic
		\end{itemize}
\end{enumerate}
}
{
\begin{itemize}
	\item Any programming language
	\item Base web technologies knowledge
	\item Any DB system knowledge
	\item Eager to learn new technologies
\end{itemize}
}
{Dominik Markiewicz}
{1 semester}
{2-6}
\end{project}
%%%%%%%%%%%%%%%%%%%%%%%%%%%%%%%%%%%%%%%%%%%%%%%%%%%%%%%%%%%%%%%%%%%%%%%%%%%%%%%%%%%%%%
\begin{project}
{Cross-applications shortcuts as a web component}
{When many web services are operated and advertised by one entity (department, company, whatever) it is wise to have consistent way to easily move user bwetween applications. Good example are Google web apps or Microsoft web apps, where it's always obvious how to jump between services in given company portfolio - by using same looking shortcuts button in every application. The goal of the project is to have web-based service that would allow for creation, maangement and display of such common component for consistent linking to many web applicatiions/pages.} 
{
Minimal finished project allows for:
\begin{itemize}
	\item Separate web application where one can 
		\begin{itemize}
			\item create new apps - with their icons and links
			\item order or position of particlar application on applications list
		\end{itemize}
	\item Web component in any technology, that can be embedded in navbar of any application, and when clicked will display list of applications user can jump to with clickable links/anchors.
\end{itemize}
Possible extension: created app could monitor health of linked applications and disable/enable or modify view of the links displayed depending on the status of linked application (unresponsive, maintanance or similar).
}
{-}
{Mateusz Wronski, Dominik Markiewicz}
{1 semester}
{4}
\end{project}
%%%%%%%%%%%%%%%%%%%%%%%%%%%%%%%%%%%%%%%%%%%%%%%%%%%%%%%%%%%%%%%%%%%%%%%%%%%%%%%%%%%%%%
\begin{project}
{Mobile phone DevOps alarms delivery}
{Develop a tool that will notify person via android app that some system or web application has crashed and/or behaves weirdly.} 
{
Application that has following components
\begin{itemize}
	\item Backend app:
		\begin{itemize}
			\item monitors web services and generates and closes alarms when web applications are not responsive or has failing healthchecks
			\item serves the mobile phone app
		\end{itemize}
	\item Android/mobile phone application:
		\begin{itemize}
			\item Allows for subscription to alarms from particular services
			\item Receives and displays alarms and alarm cancellations
			\item Allows the technical stuff to ``claim`` the alarm (``I`m working on it`` notification)
		\end{itemize}
\end{itemize}
\bigbreak
Extra: application can be notified about events/alarms from Sensu system.

Extra: backend can be notified about application events (``App x is rebooting for upgrade. Est downtime 30 minutes``)
}
{Any programming language, minimal frontend dev experience (frontend frameworks) or eager to learn Angular 2.}
{Mateusz Wronski}
{1 semester}
{3-5}
\end{project}
%%%%%%%%%%%%%%%%%%%%%%%%%%%%%%%%%%%%%%%%%%%%%%%%%%%%%%%%%%%%%%%%%%%%%%%%%%%%%%%%%%%%%%
\begin{project}
{SSL/TLS support in Facebook Presto big data drivers}
{Extend Facebook Presto MongoDB and Cassandra drivers with SSL/TLS connectivity.}
{
Presto is a distributed-query engine from Facebook that allows cross-querying different databases.

Connection to particular database implementation is handled by Connectors. Presto connectors lack support for SSL connections to databases, that would allow for connections between Presto and databases via non-separated, even public network. At the moment such configuration requires setup of many SSH tunnels.
}
{Java, basic knowledge of PKI/SSL mechanisms}
{Mateusz Wronski, Michael Dec}
{1 semester}
{1-2}
\end{project}
%%%%%%%%%%%%%%%%%%%%%%%%%%%%%%%%%%%%%%%%%%%%%%%%%%%%%%%%%%%%%%%%%%%%%%%%%%%%%%%%%%%%%%
